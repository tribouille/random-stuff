\section{Expériences}
\cventry{01/18--12/18}{Ingénieur R\&D}{Alten}{Toulouse}{Temps complet}{Développement du logiciel interne de gestion centralisée des outillages et de suivi de production de l’usine de Saint-Éloi.}

\cventry{05/17--01/18}{Création d’entreprise}{}{Télétravail}{Temps complet}{Création de ma micro entreprise dans le développement applicatif.}

\cventry{06/15--05/17}{Ingénieur R\&D}{Mediaarea.net}{Télétravail}{Temps complet}{Développement logiciel d'un validateur de standart vidéo. Les principales tâches furent la création du logiciel Open-Source MediaConch, réalisation de tests (unitaires et comportementales) et la documentation.}

\cventry{09/13--06/15}{Ingénieur R\&D Junior}{Intersec}{La Défense}{Temps complet}{Développement logiciel d'un SMS-C et d'une gateway SMS trans-protocol. Principale tâche fut la mise-à-jour du protocol GSM MAP, réalisation des tests (unitaires et comportementales) et la documentation.}

\cventry{03/13--08/13}{Ingénieur R\&D Junior}{Smartjog}{Ivry-sur-Seine}{}{Création d’un outil de streaming en Python. Utilisation de FFMPEG et de ses outils. Création d’un décodeur pour le containeur Smooth Streaming.}

\cventry{11/10--08/11}{Ingénieur R\&D Junior}{Fineduc}{Paris}{Mi-Temps puis Temps complet}{R\&D sur les problématiques d'interaction entre les technologies de streaming multimédia et les infrastructures réseaux classiques.}

\closesection
