\section{Expériences}
\cventry{09/13--06/15}{Ingénieur R\&D Junior}{Intersec}{La Défense}{Temps complet}{Développement logiciel d'un SMS-C et d'une gateway SMS trans-protocol. Principale tâche fut la mise-à-jour du protocol GSM MAP, réalisation des tests (unitaires et comportementales) et la documentation.}

\cventry{03/13--08/13}{Ingénieur R\&D Junior}{Smartjog}{Ivry-sur-Seine}{}{Création d’un outil de streaming en Python. Utilisation de FFMPEG et de ses outils. Création d’un décodeur pour le containeur Smooth Streaming.}

\cventry{05/11--08/11}{Ingénieur R\&D Junior}{Fineduc}{Paris}{Temps complet}{R\&D sur les problématiques d'interaction entre les technologies de streaming multimédia et les infrastructures réseaux classiques.}

\cventry{11/10--05/11}{Ingénieur R\&D Junior}{Fineduc}{Paris}{Temps partiel}{}

\cventry{10/09--12/09}{Ingénieur R\&D Junior}{ACECOR-COTEP}{Paris}{}{Développement d'un système de géo-localisation sur système embarqué, en C.}

\cventry{10/09}{Participation au Ports Hackathon 2009 d'OpenBSD (P2K9)}{}{Hongrie}{}{Un Ports Hackathon est un regroupement de développeurs internationaux qui s’occupe des Ports (plus ou moins l’équivalent des packages sous Linux). Ce regroupement permet de rencontrer des les autres développeurs et de partager nos problématiques. Durant cette semaine, j'ai réalisé plusieurs patchs, tels que : Terminale, subtitleeditor...}

\cventry{07/09--09/09}{Développeur Web}{ACECOR-COTEP}{Paris}{}{Développement d'une application Web permettant l'organisation des départs et arrivées des trains de la SNCF. Code en PHP, HTML/CSS et javascript.}
