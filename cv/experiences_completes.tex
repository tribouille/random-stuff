\section{Expériences}
\cventry{05/2011--08/2011}{Ingénieur R\&D Junior}{Fineduc}{Paris}{Temps complet}{R\&D sur les problématiques d'interaction entre les technologies de streaming multimédia et les infrastructures réseaux classiques. %
 \begin{itemize}\setlength\itemindent{6pt}
   \item Implémentation du muxer MPEG-TS dans c-RTMP-server avec utilisation du codec H264 pour la vidéo et MPEG-3 pour le son
   \item Ce serveur permet d’enregistrer la vidéo-conférence et de streamer en MPEG-TS ou en RTMP, avec un flux passant de 3M/s à 220k/s
   \item Développement en C++
   \item Compréhension de la notion de temps-réel
   \item Reprise de code existant afin d’y implémenter de nouvelles fonctionnalités
 \end{itemize}
}

\cventry{11/2010--05/2011}{Ingénieur R\&D Junior}{Fineduc}{Paris}{Temps partiel}{ %
  \begin{itemize}\setlength\itemindent{6pt}
    \item Implémentation de l’utilisation du codec H264 dans les modules Imem et Smem de la bibliothèque VLC
    \item Réalisation d’un plugin à l’aide de la bibliothèque Firebreath, écrite en C++, permettant de récupérer un flux vidéo (d’une webcam) et de le streamer grâce à la bibliothèque VLC
   \item Développement en C++
    \item Discussion avec l’un des développeur principal de VLC afin de pouvoir comprendre et réaliser ce projet
  \end{itemize}
}

\cventry{10/2009--12/2009}{Ingénieur R\&D Junior}{ACECOR-COTEP}{Paris}{}{Développement d'un système de géo-localisation sur système embarqué, en C.%
  \begin{itemize}\setlength\itemindent{6pt}
    \item Reprise du code existant et adaptation aux nouvelles infrastructures
    \item Portage du projet d’un système SCO-Unix à un Ubuntu-Server
    \item Configuration du matériel d'envoi d’information GPS par GPRS
    \item Projet réalisé en autonomie, avec reprise du code existant
 \end{itemize}
}

\cventry{10/2009}{Participation au Ports Hackathon 2009 d'OpenBSD (P2K9)}{}{Hongrie}{}{Un Ports Hackathon est un regroupement de développeurs internationaux qui s’occupe des Ports (plus ou moins l’équivalent des packages sous Linux). Ce regroupement permet de rencontrer des les autres développeurs et de partager nos problématiques. Durant cette semaine, j'ai réalisé plusieurs patchs, tels que : Terminale, subtitleeditor...}

\cventry{07/2009--09/2009}{Développeur Web}{ACECOR-COTEP}{Paris}{}{Développement d'une application Web permettant  l'organisation des départs et arrivées des trains de la SNCF. Code en PHP, HTML/CSS et javascript.}
