\section{Expériences Professionnelles}
\cventry{01/18--12/18}{Ingénieur R\&D}{Alten}{Toulouse}{Temps complet}{Développement logiciel de l’outil informatique de gestion centralisée des outillages et du suivi de production de l’usine de Saint-Éloi (création de pièces, aléas, etc). %
 \begin{itemize}\setlength\itemindent{6pt}
   \item Amélioration/maintenance du logiciel de suivi et de centralisation des outillages.
   \item Création d’un module de durée de vie des outillages en fonction du plan de production.
   \item Assistant de planification des pièces, avec la mise en relation des différents services.
   \item Création d’un module de suivi, en temps réel, de l’avancement de la production.
   \item Développement en VB.NET principalement sur la partie back-end, WPF (XAML) pour la partie Front-end avec une base de donnée MS SQL.
   \item Communication avec les différents outils de l’usine pour récupérer les informations, créer des indicateurs, des réunions, etc (Excel, SAP, Outlook)
 \end{itemize}
}

\cventry{06/15--05/17}{Ingénieur R\&D}{Mediaarea.net}{Télétravail}{Temps complet}{Développement logiciel d'un validateur de standards vidéo. %
 \begin{itemize}\setlength\itemindent{6pt}
   \item Amélioration/maintenance du Codec vidéo FFV1.
   \item Création du logiciel MediaConch en C++/Qt.
   \item Validation utilisant du XML et la libxml.
   \item Écriture de tests unitaires (C++) et comportementales (Python).
   \item Création de vidéos pour benchmark du logiciel.
   \item Développement en C++ principalement sur différentes features du produits (règles de routage, filtrage, protocolaires)
 \end{itemize}
}

\cventry{09/13--06/15}{Ingénieur R\&D Junior}{Intersec}{La Défense}{Temps complet}{Développement logiciel d'une plateforme faisant office de SMS-C et de gateway SMS trans-protocol. %
 \begin{itemize}\setlength\itemindent{6pt}
   \item Développement en C principalement sur différentes features du produits (règles de routage, filtrage, protocolaires)
   \item Rework et amélioration des couches GSM SMS et GSM MAP.
   \item Amélioration/maintenance des couches SCCP/TCAP.
   \item Écriture de tests unitaires (C) et comportementales (Python).
   \item Création d'un simulateur pour générer un réseau GSM.
 \end{itemize}
}

\section{Stages}

\cventry{03/2013--08/2013}{Ingénieur R\&D Junior}{Smartjog}{Ivry-sur-Seine}{}{Création d’un outil de streaming en Python. Utilisation de FFMPEG et de ses outils. %
 \begin{itemize}\setlength\itemindent{6pt}
   \item Implémentation et développement d’un module dans Nginx-rtmp pour encoder en Smooth Streaming
   \item Développement d’un décodeur Smooth Streaming dans FFmpeg
   \item Développement en C
 \end{itemize}
}

\cventry{05/2011--08/2011}{Ingénieur R\&D Junior}{Fineduc}{Paris}{Temps complet}{R\&D sur les problématiques d'interaction entre les technologies de streaming multimédia et les infrastructures réseaux classiques. %
 \begin{itemize}\setlength\itemindent{6pt}
   \item Implémentation du muxer MPEG-TS dans c-RTMP-server avec utilisation du codec H264 pour la vidéo et MPEG-3 pour le son
   \item Développement en C++ et C
   \item Compréhension de la notion de temps-réel
   \item Reprise de code existant afin d’y implémenter de nouvelles fonctionnalités
 \end{itemize}
}

\cventry{11/2010--05/2011}{Ingénieur R\&D Junior}{Fineduc}{Paris}{Temps partiel}{ %
  \begin{itemize}\setlength\itemindent{6pt}
    \item Implémentation de l’utilisation du codec H264 dans les modules Imem et Smem de la bibliothèque VLC
    \item Réalisation d’un plugin à l’aide de la bibliothèque Firebreath, écrite en C++, permettant de récupérer un flux vidéo (d’une webcam) et de le streamer grâce à la bibliothèque VLC
   \item Développement en C++
    \item Discussion avec l’un des développeur principal de VLC afin de pouvoir comprendre et réaliser ce projet
  \end{itemize}
}

\cventry{10/2009--12/2009}{Ingénieur R\&D Junior}{ACECOR-COTEP}{Paris}{}{Développement d'un système de géo-localisation sur système embarqué, en C.%
  \begin{itemize}\setlength\itemindent{6pt}
    \item Reprise du code existant et adaptation aux nouvelles infrastructures
    \item Portage du projet d’un système SCO-Unix à un Ubuntu-Server
    \item Configuration du matériel d'envoi d’information GPS par GPRS
    \item Projet réalisé en autonomie, avec reprise du code existant
 \end{itemize}
}

\cventry{10/2009}{Participation au Ports Hackathon 2009 d'OpenBSD (P2K9)}{}{Hongrie}{}{Un Ports Hackathon est un regroupement de développeurs internationaux qui s’occupe des Ports (plus ou moins l’équivalent des packages sous Linux). Ce regroupement permet de rencontrer des les autres développeurs et de partager nos problématiques. Durant cette semaine, j'ai réalisé plusieurs patchs, tels que : Terminale, subtitleeditor...}

\cventry{07/2009--09/2009}{Développeur Web}{ACECOR-COTEP}{Paris}{}{Développement d'une application Web permettant  l'organisation des départs et arrivées des trains de la SNCF. Code en PHP, HTML/CSS et javascript.}
\closesection
